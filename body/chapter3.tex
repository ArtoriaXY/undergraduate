\section{危险因素辨识与评价}

\subsection{危险因素辨识目的和范围}

根据规范《重大危险源辨识》(GBl821 8—2000),所谓重大危险源就是指长期或临时地生产、加工、搬运或贮存危险物质,
且危险物质的数量等于或大于临界量的单元对于路桥工程施工过程中,对于重大危险源的界定主要包括:
人的危险行为及管理的漏洞、物的不安全的状态、恶劣的环境影响等,由于建筑施工现场的复杂性,工程施工事故可能随时发生,
并可导致人员死亡及伤害、破坏、财产损失,这对于建筑施工工程的整体施工进度已经企业的经济效益都会造成恶劣的影响,
甚至危及企业的发展,因此建筑施工过程中对于重大危险源的辨识、评价和控制,就显得格外重要。

对于重大危险源的辨识,可根据对危险源危险等级的评定方法进行,一般是对施工过程中危险源带来的风险进行评价分析,
根据评价结果又针对性地进行风险控制,从而达到持续改进的目的。常用的风险评价方法有:
作业条件危险性评价法(LEC)、矩阵法、预先危害分析(PHA)、故障类型及影响分析(FMEA)、风险概率评价法(PRA)、危险可操作性研究(HAZOP)、事故树分析(ETA)=等等。

\subsection{危险源辨识}
\subsubsection{基坑工程危险源辨识}
\subsubsection{钢筋工程危险源辨识}
\subsubsection{模板工程危险源辨识}
\subsubsection{混凝土工程危险源辨识}
\subsubsection{脚手架工程危险源辨识}
\subsubsection{砌体工程危险源辨识}
\subsubsection{吊装作业危险源辨识}

\quan{1} 起重机在运行中对人体造成的挤压或撞击。

\quan{2} 起重机吊钩超载断裂、吊运时钢丝绳从吊钩中滑出。

\quan{3} 吊运中重物坠落造成物体打击,重物从空中落到地面又反弹伤人。

\quan{4} 钢丝绳或麻绳断裂造成重物下落;使用应报废的钢丝绳,使用的吊具吊运超过额定起重量的重物等造成重物下落。

\quan{5} 汽车起重机作业场所地面不平整、支撑不稳定、配重不平衡、重物超过额定起重量而造成起重机倾覆。

\quan{6} 风力过大、违章作业造成起重机倾覆。

\quan{7} 机械传动部分未加防护,造成机械伤害;违章在卷扬机钢丝绳上面通过,运动中的钢丝绳将人挤伤或绊倒。

\quan{8} 载货升降机违章载人。

\quan{9} 人站在起重臂下等危险区域。

\quan{10} 电气设备漏电、保护装置失效、裸导线未加屏蔽等造成触电。

\quan{11} 人站立或坐在吊钩上。

\quan{12} 吊运时无人指挥、作业区内有人逗留、运行中的起重机的吊具及重物摆动撞击行人。

\quan{13} 起重工及其他操作人员未戴安全帽等个人防护用品。

\quan{14} 司机视野不清、与指挥人员联络不畅,或误解吊运信号。

\quan{15} 吊挂方式不正确,造成重物从吊钩中脱出。

\quan{16} 使用的钢丝绳超过安全系数。

\quan{17} 钢丝绳从滑轮中跳出轮槽。

\quan{18} 制动器出现裂纹、摩擦垫片磨损过多。

\subsubsection{其他工程危险源辨识}
\subsection{安全评价}
\subsubsection{评价依据}
\subsubsection{评价目的与评价范围}
\subsubsection{安全评价方法}
\subsubsection{评价单元的划分}
\subsubsection{基坑坍塌事故故障树法安全分析}
\subsubsection{模板工程坍塌事故故障树法安全分析}
\subsubsection{高处坠落事故故障树法安全分析}
\subsubsection{物料提升机与施工升降机安全检查表法安全分析}
\subsubsection{施工用电安全检查表法安全分析}
\subsubsection{脚手架工程预先危害分析法安全分析}