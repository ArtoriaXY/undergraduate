\pagenumbering{arabic}
\section{工程概况}
\subsection{施工组织设计编制基本原则}

施工组织设计按照编制对象,可分为施工组织总设计、单位工程施工组织设计。施工组织设计应包括编制依据、工程概况、施工部署、施工准备与资料配置计划、施工进度计划、主要施工方法、施工管理措施、施工现场平面布置等主要内容;施工组织设计的编制必须遵循工程建设程序,并符合下列原则:

\begin{itemize}

    \item [1)] 符合国家有关法律法规、现行规范,符合地方规程、行业标准的要求;

    \item [2)] 满足建筑施工合同或招标文件中关于建筑工程进度、质量、环境保护、职业健康、安全、工程造价等工程管理目标的要求;

    \item [3)] 积极开发、推广运用新技术、新工艺、新材料、新设备;

    \item [4)] 坚持科学的施工程序和合理的施工顺序,做到资源的优化组织和合理配置,采用流水施工和网络计划的方法,实现均衡施工,努力实现科学、合理的经济技术指标;

    \item [5)] 积极响应国家关于低碳、节能、环保方面的方针、政策;采取先进的技术和管理措施,推广建筑节能和绿色施工。

    \item [6)] 与建筑施工单位质量、环境、职业健康安全、项目管理规范四合一标准的有效结合,贯彻质量、环境、职业健康安全管理国家管理规范的要求;

\end{itemize}

\subsection{施工组织设计编制程序}
\subsection{指导方针及编制依据}

\subsubsection{指导方针}
\subsubsection{编制依据}

施工组织设计应以下列内容为主要编制依据:

\begin{itemize}

    \item [1)] 与建筑工程有关的法律、法规和相关文件;

    \item [2)] 国家现行有关标准、规范和技术经济指标;

    \item [3)] 工程所在地的行政主管部门的管理要求;

    \item [4)] 建筑施工行业相关的的质量、环境、职业健康安全管理体系管理规范的要求;

    \item [5)] 工程施工合同及招投标文件;

    \item [6)] 工程设计文件

    \item [7)] 项目周边环境、现场条件、工程地质和水文、气象等自然条件;

    \item [8)] 与工程项目施工有关的资源供应、生产要素配置情况;

    \item [9)] 施工单位的生产能力、机具设备状况、技术水平等等。
    
\end{itemize}

\subsection{工程概况}
\subsection{建筑设计概况}
\subsection{结构设计概况}
\subsection{气象地质特点}

