\section{施工现场安全管理措施}

\subsection{安全管理方针及目标}
\subsubsection{安全管理方针}

(1) 杜绝人身重伤、死亡责任事故,轻伤负伤率指标控制在 0.3\%;

(2) 确保安全文明标准化工地;

(3) 安全隐患排查整改率 100\%;

(4) 实现无重大设备事故、无重大火灾事故、无重大交通事故、无多人急性中毒事故;

(5) 认真贯彻中国建筑《安全生产管理手册》和《施工现场安全防护标准图册》,施工现场安全防护标准化合格率 100\%;

(6) 安全管理人员持证上岗率 100\%,特种作业管理人员持证上岗率 100\%。

\subsection{现场特种作业安全教育}

根据《安全生产法》、《建筑安全生产管理条例》、国家安全生产监督局第13号令《特种作业人员安全技术培训考核管理办法》,
特种作业人员必须经过专门培训,考试合格获得上岗证后方可进行特种作业操作。\\

(1) 电工安全教育\\

\quan{1} 作业人员,必须经专业技术培训、考试合格,方准上岗独立操作;

\quan{2} 必须穿绝缘鞋,戴好安全帽及防护手套和工作相关的防护用具。携带试电笔,不准使用无绝缘的金属工具,以免造成导线接地,短路及人身触电事故。
工作时严禁用手触摸带电设备及非绝缘部分;

\quan{3} 在施工现场检修机械设备或线路,必须在配电室拉掉相应的控制开关,并悬挂“严禁合闸”的警示牌;

\quan{4} 巡视线路或检修配电子表箱,不论线路是否停电,均视为带电,当发现故障时,应有防止跨步电压及接触电压的措施。
方可进行检修工作;

\quan{5} 对施工现场的配电箱内的漏电保护器要每周进行一次检查;\\


(2) 焊工安全教育\\

\quan{1} 焊接作业人员,必须经专业技术培训、考试合格,方准上岗独立操作。非焊工严禁进行电焊作业;

\quan{2} 操作前应检查所有工具、电焊机、电源开关及线路是否良好,金属外壳应有安全可靠的接地,进出应有完整的防护罩;

\quan{3} 操作时应穿工作服、绝缘鞋和戴电焊手套、防护面罩等安全防护用品。清除作业点周围 10 米范围内的易燃易爆物品;

\quan{4} 高处作业时,必须专人监护、系好安全带,焊工必须站在稳固的操作平台上作业,严禁将焊接电缆挂在脖颈上;

\quan{5} 焊接时二次线必须双线到位,严禁借用金属管道,金属、轨道及结构钢筋作回路地线;

\quan{6} 清除焊渣时,应佩带眼睛式面罩,防止焊渣溅入眼内或烫伤皮肤;

\quan{7} 搬运氧气瓶、乙炔气瓶时,必须装好防振圈,避免碰撞、振动;使用保管中应避免曝晒和火烤,乙炔瓶、氧气瓶及易燃物品等严禁同车运输;

\quan{8} 气瓶使用时,氧气瓶与乙炔气瓶之间不应小于 5m。 氧气瓶、乙炔瓶与切割点、明火安全距离 10 米以上;

\quan{9} 氧气瓶及压力表的部位,均不得沾染油污,不准撞击、滚动和曝晒。氧气表和乙炔表冻结时,不准用火烤或锤打,应使用热水或蒸汽解冻。 \\

(3) 铲车司机安全教育\\

\quan{1} 作业前应检查发动机的油、水应加足,各操纵杆放在空挡位置,制动灵敏可靠,灯光仪表齐全、有效方可起动;

\quan{2} 机械起动必须先鸣笛,将铲斗提升离地面 50cm 左右。行驶中可用高速档,但不得进行升降和翻转铲斗动作,作业时应使用低速档,铲斗下方严禁有人,严禁用铲斗载人;

\quan{3} 装载机不得在倾斜的场地上作业,作业区不得有障碍物及无关人员。装卸作业应在平整地面进行;

\quan{4} 向汽车内卸料时,严禁将铲斗从驾驶室顶上越过,铲斗不得碰撞车厢,严禁车厢内有人,不得用铲斗运物料;

在沟槽边卸料时,必须设专人指挥,装载机前轮应与沟槽边缘保持不少于 2m 的安全距离,并放置挡木挡掩;

\quan{5} 将大臂升起进行维护、润滑时,必须将大臂支撑稳固;

\quan{6} 作业后应将装载机开至安全地区,不得停在坑洼积水处,必须将铲斗平放在地面上,将手柄放在空挡位置,拉好手制动器。关闭门窗加锁后,方可离开。\\


(4) 挖掘机机操作员安全教育\\

\quan{1} 挖掘机司机必须经专业技术培训,考试合格取证后方可上
车独立操作。司机应熟知挖掘机的机械原理,保养规则,安全操作规程,并要按规定严格执行。严禁酒后或身体不适时进行操作;

\quan{2} 挖掘机在工作前,应向施工人员了解施工条件和任务。挖掘机进入现场后,司机应遵守施工现场的有关安全规则;

\quan{3} 挖掘机在工作前,应按照日常例行保养项目,对挖掘机进行检查、保养、调整、紧固。挖掘机在工作中,严禁进行维修、保养、紧固等工作。工作过程中若发生异响、异味、温升过高等情况,应立即停车检查;

\quan{4} 挖掘机工作时应当处于水平位置,并将走行机构刹住。若地面泥泞、松软和有沉陷危险时,应用枕木或木板垫妥;

\quan{5} 若必须在挖掘机回转半径内工作时,挖掘机必须停止回转,并将回转机构刹住后,方可进行。同时,机上机下人员要彼此照顾,密切配合,确保安全;

\quan{6} 挖掘机装载活动范围内,不得停留车辆和行人。若往汽车上卸料时,应等汽车停稳,驾驶员离开驾驶室后,方可回转铲斗,向车上卸料。挖掘机回转时,
应尽量避免铲斗从驾驶室顶部越过。卸料时,铲斗应尽量放低,但又注意不得碰撞汽车的任何部位;

\quan{7} 挖掘机回转时,应用回转离合器配合回转机构制动器平稳转动,禁止急剧回转和紧急制动;

\quan{8} 铲斗未离开地面前,不得做回转、走行等动作。铲斗满载悬空时,不得起落臂杆和行走;

\quan{9} 挖掘机不论是作业或走行时,都不得靠近架空输电线路。如必须在高低压架空线路附近工作或通过时,机械与架空线路的安全距离,必须符合规定尺寸。雷雨天气,严禁在架空高压线近旁或下面工作。
在地下电缆附近作业时,必须查清电缆的走向,并用白粉显示在地面上,并应保持1米以外的距离进行挖掘;

\quan{10} 夜间工作时,作业地区和驾驶室,应有良好的照明;

\quan{11} 挖掘机工作后,应将机械驶离工作地区,放在安全、平坦的地方。将机身转正,使内燃机朝向阳方向,铲斗落地,
并将所有操纵杆放到“空档”位置,将所有制动器刹死,关闭发动机。按照保养规程的规定,做好例行保养。关闭门窗并上锁后,方可离开;

\quan{12} 挖掘机装卸车时,应由专人指挥。装卸过程中,挖掘机在坡道上严禁回转或转向。装车时若发生危险情况,可将铲斗放下,协助制动,然后挖掘机缓缓退下。\\


(5) 起重机械操作员安全教育\\

\quan{1} 起重指挥应由技术熟练、懂得起重机械性能、经过专业培训的人员担任,持证上岗。指挥时应站在能够照顾到全面工作的地点,所发信号应事先统一,并做到准确、清楚;

\quan{2} 所有人员严禁在起重臂和吊起的重物下面停留或行走;

\quan{3} 起吊物件应使用交互捻制的钢丝绳。钢丝绳如有扭结、变形,断丝、锈蚀等异常现象,应及时降低使用标准或报废;

\quan{4} 钢丝绳中有一整股折断时、断丝数目在使用中增加很快时,该绳应立即更换。钢丝绳有明显的内部腐蚀时、钢丝绳局部外层钢丝伸长呈笼形态时,应立即报废。\\

\subsection{项目安全生产管理组织机构}

\subsection{安全生产管理制度}
\subsubsection{安全教育制度}

\subsubsection{消防安全管理制度}

\subsubsection{特种设备安全生产管理制度}

\subsubsection{高空作业安全管理制度}

(1) 高空作业管理目标\\

安全生产目标是企业经济指标的重要组成部分,根据《安全生产法》等法律法规和行业安全管理标准,制订安全生产目标管理制度。

安全与生产、安全与效益是一个整体,当发生矛盾时,必须坚持“安全第一”的原则,遵守职业健康安全法律法规,积极为员工创造适宜的、良好的工作环境,以保护员工的身心健康和职业卫生;
为有效地消除和控制危害,需要建立本质安全的科学观念,预防是最佳的选择。需要推行科学的管理体系,建立安全标准化,实行风险预防型管理,
积极采用先进的技术、工艺和设计,树立所有意外事故和职业病都是可以预防的观念;
安全生产的保障需要人机环境的安全系统协调,从人机环境的综合治理入手,坚持不懈、持续改进,没有最好,只有更好。建立安全标准化,也建立了安全工作的长效机制。\\

(2) 高空作业安全管理规定\\

\quan{1} 审批人员赴高处作业现场,检查确认安全措施后,方可批准高处作业。 从事高处作业的单位必须进行高处作业风险分析,落实安全防护措施,方可施工;

\quan{2} 高处作业人员必须经安全教育,熟悉现场环境和施工安全要求。对患有职业禁忌症和年老体弱、疲劳过度、视力不佳及酒后人员等,不准进行高处作业;

\quan{3} 高处作业前,作业人员应检查确认安全措施落实后,方可施工,否则有权拒绝施工作业;

\quan{4} 高处作业人员要按照规定穿戴劳动保护用品,作业前要检查、作业中要正确使用防坠落用品与登高器具、设备; 

\quan{5} 高处作业应设监护人对高处作业人员进行监护,监护人应坚守岗位。高处作业前,施工单位要制订安全措施;

\quan{6} 不符合高处作业安全要求的材料、器具、设备不得使用。
高处作业所使用的工具、材料、零件等必须装入工具袋,上下时手中不得持物;不准投掷工具、材料及其他物品;易滑动、易滚动的工具、材料堆放在脚手架上时,应采取措施,防止坠落。\\

(3) 高空作业分级\\

凡在离地面两米以上进行的作业,都属于高空作业;\\

\quan{1} 作业高度在2米至5米时,称为一级高处作业;

\quan{2} 作业高度在5米以上至15米时,称为二级高处作业;

\quan{3} 作业高度在15米以上至30米时,称为三级高处作业;

\quan{4} 作业高度在30米以上时,称为特级高处作业。\\

\subsection{安全文明施工措施}
\subsubsection{文明施工管理}

项目文明施工是指保持施工场地整洁、卫生,施工组织科学,施工程序合理的一种施工活动。
实现文明施工,不仅要着重做好现场的场容管理工作,而且还要相应做好现场材料、
设备、安全、技术、保卫、消防和生活卫生等方面的管理工作。
一个工地的文明施工水平是该工地乃至所在企业各项管理工作水平的综合体现。

首先,工程项目在施工准备阶段先编制单位工程《施工组织设计》
,《施工组织设计》中必须编制现场文明施工措施,并经相关部门审核、审批合格后方能组织实施。\\

文明施工应遵守以下项目:

(1) 施工现场要建立文明施工责任制,划分区域,明确管理负责人,实行挂牌制,做到现场清洁整齐;

(2) 施工现场场地平整,道路坚实畅通,有排水措施,基础、地下管道施工完后要及时回填平整,清除积土;临时水电要有专人管理,不得有长流水、长明灯;

(3) 施工现场的临时设施,包括生产、办公、生活用房、仓库、料场、临时上下水管道以及照明、动力线路,要严格按施工组织设计确定的施工平面图布置、搭设或埋设整齐;

(4) 工人操作地点和周围必须清洁整齐,做到活完脚下清,工完场地清,丢洒在楼梯、楼板上的杂物和垃圾要及时清除。严禁损坏污染成品,堵塞管道;

(5) 施工现场不准乱堆垃圾及余物。应在适当地点设置临时堆放点,并定期外运。清运垃圾及流体物品,要采取遮盖防漏措施,运送途中不得遗撒。
建筑物内清除的垃圾渣土,要通过临时搭设的竖井或利用电梯井或采取其他措施稳妥下卸,严禁从门窗口向外抛掷;

(6) 根据工程性质和所在地区的不同情况,采取必要的围护和遮挡措施,并保持外观整洁;

(7) 施工现场严禁居住家属,严禁居民、家属、小孩在施工现场穿行、玩耍;

(9) 施工现场应建立不扰民措施,针对施工特点设置防尘和防噪声设施,夜间施工必须有当地主管部门的批准。

\subsubsection{环境保护}

为加强建设工程施工现场管理,防止因建筑施工对环境的污染,依据《中华人民共和国环境保护法》等有关规定制定本措施,意在在施工过程中有效防治扬尘、
噪声、固体废物和废水等污染环境的情况。根据相关规定施工现场建立环境保护管理体系,责任落实到人,并保证有效运行。
应对施工现场防治扬尘、噪声、水污染及环境保护管理工作进行检查和对职工进行环保法规知识培训考核。按照现场常见的环境污染类别,可将环境保护规定大致分为以下三类:\\

(1) 防治大气污染:\\

\quan{1} 施工现场应采取覆盖、固化、绿化、洒水等有效措施,做到不泥泞、不扬尘。

\quan{2} 遇有四级风以上天气不得进行土方回填、转运以及其他可能产生扬尘污染的施工。  

\quan{3} 施工现场有专人负责环保工作,配备相应的洒水设备,及时洒水,减少扬尘污染。  

\quan{4} 建筑物内的施工垃圾清运必须采用封闭式容器吊运,严禁凌空抛撒。施工现场设密闭式垃圾站,施工垃圾、生活垃圾分类存放。施工垃圾清运时提前适量洒水,并按规定及时清运消纳。  

\quan{5} 土方、渣土和施工垃圾的运输,必须使用密闭式运输车辆,并与持有消纳证的运输单位签定防遗撒、扬尘、乱倒协议书。施工现场出入口处设置洗车池。  

\quan{6} 施工现场混凝土浇注使用预拌混凝土,施工现场装修阶段设置搅拌机的机棚必须封闭,并配备有效的降尘防尘装置。  

\quan{7} 拆除旧有和大临建筑时,随时洒水,减少扬尘污染。渣土要在拆除施工完成之日起三日内清运完毕,并遵守拆除工程的有关规定。  \\

(2) 防治水污染:\\

\quan{1} 搅拌机、混凝土输送泵及运输车辆清洗处设置二级沉淀池,废水不得直接排入市政污水管网,经二次沉淀后用于洒水降尘。  

\quan{2} 现场存放油料、油质脱模剂,必须对库房进行防渗漏处理,储存和使用采取防泄漏措施,防止油料泄漏,污染土壤水体。\\

(3) 防治噪声污染:\\

\quan{1} 施工现场遵照《中华人民共和国建筑施工场界噪声限值》制定降噪措施。建筑施工过程中使用的设备,可能产生噪声污染的,按有关规定向工程所在地的环保部门申报。  

\quan{2} 施工现场的电锯、电刨、搅拌机、固定式混凝土输送泵、大型空气压缩机等强噪声设备搭设封闭式机棚,并尽可能设置在远离居民区的一侧,以减少噪声污染。  

\quan{3} 因生产工艺上要求必须连续作业或者特殊需要时,确需在 22 时至次日 6 时期间进行施工的,在施工前到工程所在地建设行政主管部门提出申请,经批准后方可进行夜间施工,做好周边居民工作。并公布施工期限。  进行夜间施工作业时,应采取措施,最大限度减少施工噪声,采用低噪声震捣棒等方法。
承担夜间材料运输的车辆,进入施工现场严禁鸣笛.装卸材料应做到轻拿轻放,最大限度地减少噪声扰民。  

\quan{4} 施工现场进行噪声值监测,监测方法执行《建筑施工场界噪声测量方法》,噪声值不超过国家或地方噪声排放标准。  

\subsection{安全保障措施}
\subsubsection{基坑工程安全保障措施}

\subsubsection{钢筋工程安全保障措施}

\subsubsection{模板工程安全保障措施}

\subsubsection{混凝土工程安全保障措施}

\subsubsection{砌筑工程安全保障措施}

\subsubsection{脚手架搭拆工程安全保障措施}

\subsubsection{施工用电安全保障措施}

\subsubsection{施工机具安全保障措施}
