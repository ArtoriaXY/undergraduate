\begin{center}
    \section*{\zihao{-3}  \textbf{引 ~~ 言}}
\end{center}
\pagenumbering{arabic}

\vskip0.5cm
\zihao{-4}理工文科所有专业本科生的毕业设计(论文)都应有“引言”的内容。如果引言部分省略,该部分内容在正文中单独成章,标题改为文献综述,用足够的文字叙述。从引言开始,是正文的起始页,页码从1开始顺序编排。
针对做毕业设计:说明毕业设计的方案理解,阐述设计方法和设计依据,讨论对设计重点的理解和解决思路。

针对做毕业论文:说明论文的主题和选题的范围;对本论文研究主要范围内已有文献的评述;说明本论文所要解决的问题。建议与相关历史回顾、前人工作的文献评论、理论分析等相结合。

注意:是否如实引用前人结果反映的是学术道德问题,应明确写出同行相近的和已取得的成果,避免抄袭之嫌。注意不要与摘要内容雷同。

书写格式说明:
标题“引言”选用模板中的样式所定义的“引言”;或者手动设置成字体:黑体,居中,字号:小三,1.5倍行距,段后1行,段前为0行。

引言的字数在3000字左右(毕业设计类引言可适当调整为800字左右)。引言正文选用模板中的样式所定义的“正文”,每段落首行缩进2字;或者手动设置成每段落首行缩进2字,宋体,小四,多倍行距 1.25,段前、段后均为0行,取消网格对齐选项。
\addcontentsline{toc}{section}{引言}
\pagestyle{fancy}
