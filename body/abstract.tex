\begin{center}
\section*{\zihao{-3}  \textbf{摘 ~~ 要}}
\end{center}

\vskip0.5cm
\zihao{-4}本设计名称为“河南省旅游中心项目”,建筑总高度为 32.25m, 建筑层数为 5  层,主要针对该项目的施工过程进行全面的安全方案设计。
通过制定本工程的施工组织设计,了解各个部分工程的基本施工方案,制定评价单元,从而确定施工过程中人的不安全行为和物的不安全状态,
对施工现场中的危险源进行辨识, 运用事故树、预先危害分析、安全检查表等方法对施工现场中存在的危险源进行评价, 对已经发现的危险危害因素做出预防措施,
并且制定相应的应急预案。

本工程建设地点为河南省郑州市,用地面积为 33333 平方米,总建筑面积为 29175 平方米,其中地上建筑面积 21332 平方米,
地下建筑面积为 7843 平方米,建筑占地面积 8116 平方米,建筑高度 36 米,地上五层,地下一层;本工程为一类高层
建筑,建筑耐火等级为一级,设计使用年限为 50 年。结构性质为 A 级高度高层建筑,结构体系为钢筋硂框架——剪力墙结构,
安全等级二级,主体结构采用桩基。

根据 “安全第一,预防为主,综合治理”的安全方针,建立项目安全生产管理组织机构,健全和完善相关管理制度。根据危险源辨识与评价,制定重大事故的相应应急预案,
形成完整的管理责任流程,为项目部安全管理提供完整、高效的管理依据。\\



{\zihao{-4} \heiti 关键词: 施工组织设计;危险源辨识;安全评价;专项施工方案;应急预案}
\addcontentsline{toc}{section}{摘要}
\pagestyle{fancy}

\clearpage
\begin{center}
    \section*{\zihao{-3}  \textbf{Safety Construction Organization design of Henan Tourism Center}}
    \zihao{-3} Abstract
    \end{center}

   %用了Times New Roman字体来美化观感

   \zihao{-4}This design is entitled "Henan Tourism Center safety construction or ganization and design", building a total height of 95.25 m, building layer number is 30, 
   mainly for the project construction process to conduct a comprehensive safety plan design. Are formulated by the construction organization design, 
   to understand each part project of basic construction plan, make evaluation unit, to determine the construction process of human uns afe behavior 
   and unsafe state of the content of the construction site of the hazards are identif ied, using the fault tree, preliminary hazard analysis, safety 
   check list method to evaluate th e hazards that exist in the construction site, have found that the risk of harm factors to make preventive measures,
    and formulate the corresponding contingency plans.

    The construction site of the project is Zhengzhou City, Henan Province, with a land area of 33333 square meters and a total construction area of 29175 square meters, 
    including 21332 square meters of above ground construction area, 7843 square meters of underground construction area, 8116 square meters of floor area and 36 square 
    meters of building height M, five floors above the ground and one floor underground; this project is a class I high-rise building, with fire resistance rating of 
    class I and design service life of 50 years. The structural property is A-level high-rise building, the structural system is reinforced concrete frame shear wall 
    structure, the safety level is level II, and the main structure is pile foundation.
    
    The safety level is level II, and the main structure adopts pile foundation.
   According to the safety policy of "safety first, prevention first, comprehensive management ", the project safety production management 
   organization is established, and related manage ment system is improved. According to the identification and evaluation of dangerous sources,
    the corresponding emergency plan for major accidents is formulated to form a complete management responsibility flow, providing a complete 
    and efficient management basis for t he safety management of the project department.


\textbf{\zihao{4} Key Words: Construction organization design; Hazard identification; Safety assessment; Special construction plans; The emergency response plan}
\pagestyle{fancy}
\addcontentsline{toc}{section}{Abstract}



