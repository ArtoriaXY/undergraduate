\section*{\zihao{-2} \centering 摘 ~~ 要}

\vskip0.5cm
本设计名称为“X————————————————”,建筑总高度为 95.25m, 建筑层数为 30  层,主要针对该项目的施工过程进行全面的安全方案设计。
通过制定本工程的施工组织设计,了解各个部分工程的基本施工方案,制定评价单元,从而确定施工过程中人的不安全行为和物的不安全状态,
对施工现场中的危险源进行辨识, 运用事故树、预先危害分析、安全检查表等方法对施工现场中存在的危险源进行评价, 对已经发现的危险危害因素做出预防措施,
并且制定相应的应急预案。

本工程属于框架剪力墙结构,其中脚手架工程采用落地式和悬挑式脚手架两种, 搭设高度均为 18.00m;模板工程采用木模板,支模高度为 8.95m 属于高支模;
脚手架及模板支撑体系均采用 Ф48.3×3.6 钢管;基坑达到 9.40m,采用混凝土灌注桩并配有双层锚杆支护方式。均属于超出一定规模的危险性较大的分部分项工程,
风险性极大, 因此本次设计针对以上三部分分别做出了专项方案。

根据 “安全第一,预防为主,综合治理”的安全方针,建立项目安全生产管理组织机构,健全和完善相关管理制度。根据危险源辨识与评价,制定重大事故的相应应急预案,
形成完整的管理责任流程,为项目部安全管理提供完整、高效的管理依据。\\



{\zihao{4} \heiti 关键词: } \zihao{-4}施工组织设计;危险源辨识;安全评价;专项施工方案;应急预案
\addcontentsline{toc}{section}{摘要}

\clearpage
\section*{\zihao{-2} \centering \textbf{Abstract} }
   %用了Times New Roman字体来美化观感

   This design is entitled "XX safety construction or ganization and design", building a total height of 95.25 m, building layer number is 30, 
   mainly for the project construction process to conduct a comprehensive safety plan design. Are formulated by the construction organization design, 
   to understand each part project of basic construction plan, make evaluation unit, to determine the construction process of human uns afe behavior 
   and unsafe state of the content of the construction site of the hazards are identif ied, using the fault tree, preliminary hazard analysis, safety 
   check list method to evaluate th e hazards that exist in the construction site, have found that the risk of harm factors to make preventive measures,
    and formulate the corresponding contingency plans.

   This project belongs to the frame shear wall structure, in which the scaffold project adopts fl oor type and overhanging type of scaffolding, 
   and the height of erection is 18.00m. The for mwork adopts wooden template, and the supporting height is 8.95m. Scaffolding and formw ork 
   supports system adopts Ф 48.3 x 3.6 steel tube; The foundation pit reaches 9.40m, with concrete cast-in-place pile and double-deck anchor 
   bolt support. All of them belong to sub-p rojects with greater risks than a certain scale, which have great risks. Therefore, this design 
   makes special plans for the above three parts respectively.

   According to the safety policy of "safety first, prevention first, comprehensive management ", the project safety production management 
   organization is established, and related manage ment system is improved. According to the identification and evaluation of dangerous sources,
    the corresponding emergency plan for major accidents is formulated to form a complete management responsibility flow, providing a complete 
    and efficient management basis for t he safety management of the project department.
    \\

\textbf{\zihao{4} Key Words:} Construction organization design; Hazard identification; Safety assessment; Special construction plans; The emergency response plan

\addcontentsline{toc}{section}{Abstract}




